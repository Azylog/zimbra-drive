\section{Cloud App configuration}

\subsection{Nextcloud/ownCloud}\label{==subsec:NextcloudownCloudInstallation==}
    When everything is correctly configured, the Zimbra end user creates a private account in the cloud service
    that will be paired with the Zimbra user account.
    This new cloud account inherit the Zimbra user credentials and appears in users list of Nextcloud/ownCloud Admin Interface;
    however this account is active until the Zimbra Drive app is enabled.

    Nextcloud and ownCloud have the same following configuration entries.
    In Nextcloud/ownCloud administration panel must appears a new "Zimbra Drive" entry in the left sidebar
    that redirect to the configuration view. There will be the following configurations:
    \begin{itemize}
        \item (CheckBox) \textbf{Enable Zimbra authentication back end}\\
        (Mandatory checked) On check, add a configuration in config.php that let Nextcloud/ownCloud use Zimbra Drive App class. On uncheck remove this configuration.
        \item (CheckBox) \textbf{Allow Zimbra's users to log in}\\
        (Mandatory checked) Allows Zimbra users to use Nextcloud/ownCloud with their Zimbra credentials.
        \item (InputField) \textbf{Zimbra Server}\\
        (Mandatory) Zimbra webmail host or ip.
        \item (InputField) \textbf{Zimbra Port}\\
        (Mandatory) Zimbra webmail port.
        \item (CheckBox) \textbf{Use SSL}\\
        Check if the Zimbra webmail port uses SSL certification.
        \item (CheckBox) \textbf{Enable certification verification}\\
        Disable only if Zimbra has an untrusted certificate.
        \item (InputField) \textbf{Domain Preauth Key}\\
        After Zimbra end user creates a private account with the first successful access in Zimbra Drive,
        he can log in Nextcloud/ownCloud web interface with their Zimbra credentials. 
        In Nextcloud/ownCloud web interface he will find a Zimbra icon in Apps menu
        that will open new a Zimbra webmail tab without login step.\\
        This feature works only if the Zimbra Domain PreAuth Key is copied. 
        In zimbra run the following command in order to show the desired Zimbra Domain PreAuth Key:\\
        \texttt{\# As zimbra}\\
        \texttt{zmprov getDomain example.com zimbraPreAuthKey}\\
        \texttt{\# If response is empty, generate with}\\
        \texttt{zmprov generateDomainPreAuthKey domainExample.com}\\
    \end{itemize}
