\section{Remove Cloud App}
\subsection{Nextcloud/ownCloud}\label{==subsec:NextcloudownCloudUninstallation==}

    The removal of Nextcloud/ownCloud App consists in the separate steps: Clean up and App uninstall.

\begin{comment}
WARNING:
\end{comment}
\begin{warning}
    The Clean up step will delete all the Zimbra users data from Nextcloud/ownCloud and is not reversible.
    It \textbf{requires} that the Zimbra Drive is installed and enabled.\\
    Indeed, the Zimbra Drive App can be uninstalled without the removal of Zimbra users storages
    (Clean up step can be skipped).
\end{warning}

    \textbf{Clean up}

\begin{comment}
TIP:
\end{comment}
\begin{info}
    Before starting the clean up, it's recommended to disallow Zimbra users access:
    the configuration \textbf{Allow Zimbra's users to log in} should be unchecked.
\end{info}
    The following commands delete the users created by Zimbra Drive App and clean the table containing
    references to Zimbra users (replace correctly \texttt{mysql\_pwd} and \texttt{occ\_db}):

\begin{verbatim}
cd /var/www/cloud           # Go to the OCC path
mysql_pwd='password'        # database password
occ_db='cloud'              # database name for the Nextcloud / ownCloud

# In ownCloud
user_id_column='user_id'    # column name in table oc_accounts of ownCloud
# In Nextcloud
user_id_column='uid'        # column name in table oc_accounts of Nextcloud

mysql -u root --password="${mysql_pwd}" "${occ_db}" -N -s \
    -e 'SELECT uid FROM oc_group_user WHERE gid = "zimbra"' \
    | while read uid; do \
        sudo -u www-data php ./occ user:delete "${uid}"; \
        mysql -u root --password="${mysql_pwd}" "${occ_db}" \
            -e "DELETE FROM oc_accounts WHERE ${user_id_column} = '${uid}' LIMIT 1"; \
      done
\end{verbatim}

    \textbf{App uninstall}

    In Nextcloud/ownCloud platform, the Zimbra Drive App can be removed from the Nextcloud/ownCloud Admin Interface:
    the configuration should be restored unchecking \textbf{Enable Zimbra authentication back end},
    then Zimbra Drive App must be disabled from the "Enabled Apps" tab and uninstalled from the "Disabled Apps".

    With the previous steps, the Zimbra Drive App folder (\texttt{PATH\textunderscore TO\textunderscore CLOUD/apps/zimbradrive})
    is deleted but all the users files still remains in the cloud service drive:
    any configuration or file that was not previously cleaned up, is retrieved on the reinstallation of Zimbra Drive App.
